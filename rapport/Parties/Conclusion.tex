\chapter{Conclusion}
\section*{}
Ce projet nous a permis de prendre en main la bibliothèque java GraphStream afin de l'utiliser sur des données réelles stockées en base de données.
La prise en main de cette bibliothèque n'a pas été aisée, nous n'avons trouvé que peu d'exemples d'utilisation de celle-ci. Ainsi cela nous a pris du temps afin de pouvoir avancer sur le projet.
Une fois cette dernière maîtrisée, il nous a fallu accéder aux données via un JDBC. Néanmoins, nous n'avons pas réussi à corriger un problème d'affichage que nous avons, des traits sont tracés entre la fin et le début de chaque départements/communes, ce qui pollue l'affichage des frontières.
Cependant, des mesures de sécurité prises par l'INSA nous ont empêché d'avoir accès à ces données depuis l'extérieur du réseau.De ce fait, nous avons du passer par un client graphique de type X2GO afin de pouvoir tester notre code.
Nous avons également eu des difficultés à construire nos requêtes SQL pour utiliser correctement les informations de la base de données.
Nous avons géré ces difficultés aussi bien en nous réunissant afin de résoudre de grosses difficultés, qu'en nous répartissant les tâches dans le but d'avoir une répartition équitable.
Ce projet nous à permis d'utiliser des connaissances très variées que nous avons acquise tout au long de notre cursus ASI.
\section*{}
