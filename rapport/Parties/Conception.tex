\chapter{Conception}
Avant de commencer à coder le projet il nous a fallut effectuer un tutoriel pour comprendre comment utiliser la bibliothèque GraphStream. Nous avons ainsi apprit à implémenter des nœuds, les relier entre eux, construire des graphes.
Une fois ce tutoriel effectué et acquis par chacun nous avons pu commencer à implémenter le projet. \newline
Il se compose des classes CustomGraph, GestionBD et Point, deux énumérations UniteSpatiale et UniteTemporelle et enfin notre Main pour exécuter le programme.
\newline  La classe CustomGraph est la classe permettant de créer des graphes et de les afficher. Elle est la classe principale de notre projet.Elle est composée d'un constructeur basique qui prend en entrée une GestionBD, une UniteTemporelle et une UniteSpatiale. Une méthode afficher() qui construit et affiche le graphe voulu et une méthode toString().
\newline La classe GestionBD, elle, dispose d'un constructeur qui permet une connexion à notre base de données, une méthode deconnexion() pour se déconnecter, une méthode creerVue() qui permet de crée une vue à partir d'une UniteTemporelle et une UniteSpatiale, une méthode genererRequete() qui va générer une requête suivant l'UniteTemporelle. Et les deux dernières méthodes obtenirPointsLocalisation() et obtenirPointsFrontiere() permettant d'obtenir respectivement les points de la localisation et de la frontière.
\newline  L'énumérations UniteTemporelle est composée de MOIS, JOUR, HEURE, MINUTE et l'énumération UniteSpatiale est composée de COMMUNES et DEPARTEMENTS.
\newline  Notre Main qui lance le programme est composé d'un menu qui permet de choisir les paramètres voulus et enfin fait appel à la méthode afficher() de CustomGraph afin de construire et afficher le graphe demandé.
